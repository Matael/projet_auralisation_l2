\documentclass[a4paper, 11pt]{article}

	\usepackage[utf8]{inputenc}
    \usepackage[T1]{fontenc}
    \usepackage{lmodern}
    \usepackage[french]{babel}
    \usepackage[top=3cm,right=2cm,bottom=2cm,left=2cm]{geometry}

	\title{License SPI -- Projet Auralisation}
\author{Thomas \textsc{Lechat} \and Xin \textsc{Wang} \and Mathieu \textsc{Gaborit}}
\date{Expression Scientifique et Technique -- Novembre 2012}
    
\begin{document}

	\maketitle
    
    \section{Introduction possible}
    
Le champ d'application de l'acoustique est vaste, très vaste. L'acoustique des salles est probablement une des applications les plus connues et les plus souvent citées.
Parmi les possibilités de l'acoustique vis à vis des salles, l'amélioration des "performances" acoustiques d'une salle reste celle qui revient le plus souvent dans l'imagerie populaire.
Seulement, l'étude ne s'arrête pas là et s'étend de la reproduction de conditions d'émission (simulation de transducteurs), jusqu'à la reproduction de l'influence qu'une salle donnée (existente ou non) aurait sur un son y étant diffusé. Ce procédé s'appelle l'auralisation et se base sur une réponse impulsionnelle (RI) réelle de salle (dans le cas d'une salle existante) ou bien sur une RI calculée (dans le cas d'une salle à l'étude).
Au cours de ce projet, l'objectif est de réaliser l'auralisation de plusieurs salles existantes et de comparer les signaux résultants avec le son émis directement dans la salle cible.
Tout d'abord, les approximations faites au cours du projet seront listées et expliquées ; en effet, compenser à la fois les erreurs induites par la chaines de mesure, par la chaine d'excitation et les erreurs liées au traitement des données dépasse le niveau d'une seconde année de licence
Ceci étant,  une comparaison sera menée pour déterminer l'avantage de l'utilisation  d'une RI binaurale par rapport à une RI monaurale.
Ensuite, il s'agira aussi de déterminer l'influence de la source sur la précision de la RI (les sources n'étant jamais parfaites).
Enfin, l'impact du procédé de convolution mathématique devra être qualifié voire quantifié pour en déduire l'erreur commise sur le résultat.
Pour cette dernière partie, une comparaison entre les résultats mathématiques et les enregistrements des signaux dans les salles cibles sera mise en place.


\section{Acronymes} % {{{1
Nous nous contenterons ici d'une liste d'acronymes possibles, nous choisirons celui que nous jugerons le plus représentatif du projet à la fin de celui ci :

\begin{description}
\item[COPAInS] \textbf{Co}mparaison des \textbf{P}rocédés d'\textbf{A}uralisation à l'\textbf{In}terieur des \textbf{S}alles;
\item[PLUMAu] \textbf{P}rojet de \textbf{L}icence de l'\textbf{U}niversité du \textbf{M}aine: \textbf{Au}ralisation;
\item[PETAInS] \textbf{P}rojet d'\textbf{E}tude des \textbf{T}echniques d'\textbf{A}uralisation à l'\textbf{I}nterieur  des \textbf{S}alles;
\item[PECAur] \textbf{P}rojet \textbf{E}tudiant de \textbf{C}omparaison d'\textbf{Au}ralisations;
\item[CoTAGE] \textbf{Co}mparaison des \textbf{T}echniques d'\textbf{A}uralisation et \textbf{G}estion des \textbf{E}rreurs.
\end{description}

\end{document}