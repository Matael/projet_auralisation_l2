\section{Comparaison monaural/binaural}

La perception sonore humaine est dite binaurale : c'est à dire qu'il y a deux «capteurs» (en l'occurence de chaque côté
de la tête).
Cette particularité est importante dans la perception de l'espace, en effet le volume de la tête retarde la propagation
du son tout en déformant celui-ci permettant ainsi un repérage dans le plan horizontal (avec une précsion pouvant aller
jusqu'a un degré~\cite{Vor08}). Le torse a lui aussi une influence, particulièrement pour le répérage dans le plan
vertical. Au cours des mesures pour ce projet, une tête artificielle (sans torse) a été utilisé, le repérage vertical
sera donc difficile à reproduire d'après nos mesures.


