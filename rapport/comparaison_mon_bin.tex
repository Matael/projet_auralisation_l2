\chapter{Comparaison monaural/binaural}

La perception sonore humaine est dite binaurale : c'est à dire qu'il y a deux «capteurs» (en l'occurence de chaque côté
de la tête).
Cette particularité est importante dans la perception de l'espace, en effet le volume de la tête retarde la propagation
du son tout en déformant celui-ci permettant ainsi un repérage dans le plan horizontal (avec une précsion pouvant aller
jusqu'a un degré~\cite{Vor08}). Le torse a lui aussi une influence, particulièrement pour le répérage dans le plan
vertical. Au cours des mesures pour ce projet, une tête artificielle (sans torse) a été utilisé, le repérage vertical
sera donc difficile à reproduire d'après nos mesures.

\section{Essai 1 : une chanson en salle Mersenne} % {{{1

Le premier essai approfondi est réalisé en salle Mersenne. La source est au point \textbf{Source} (voir
figure~\ref{plan_mersenne}) et le recepteur au point \textbf{P1} (la tête dans la position par défaut).

\subsection{Comparaison perceptive} % {{{2

A l'écoute, la différence entre les deux résultat (monaural et binaural) est flagrante. Alors que la position de la
source est strictement indéterminable en monaural, elle est bien identifiable en binaural. On note par ailleurs la
différence de contenu fréquentiel (et en particulier la différence de niveau) sur la
figure~\ref{comp_mon_min_zoom_150_175}.

\begin{figure}[h!]
    \centering{\includegraphics[width=13cm]{comp_mon_min_zoom_150_175.png}}
	\caption{\label{comp_mon_min_zoom_150_175}Comparaison entre les contenus fréquentiels des signaux monaural et
	binaural dans la bande 150-175Hz. On remarque que le signal monaural vient effectivement s'intercaler entre les
	2 canaux du signal binaural (en haut monaural et canal droit, en bas monaural et canal gauche). Une représentation
pleine échelle est disponible en annexe}
\end{figure}
