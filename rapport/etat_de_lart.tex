\section{Etat de l'art}

Les premiers essais se rapportant à l'auralisation ont été faits par Spandöck et al. en 1929.
Ces travaux ont été repris et améliorés après l'apparition des ordinateurs ; en utilisant cette nouvelle puissance de
calcul et, vers la fin des années 1960, le premier logiciel de simulation d'acoustique des salles fut développé
(Krokstad).

Le mot «auralisation» lui-même fut utilisé pour la première fois par Kleiner et al. dans l'article
\underline{Auralization -- An Overview}~\cite{kleiner1993}.

Dans la sommes des techniques utilisées pour aboutir à la reproduction des conditions acoustiques d'une salle, deux
reviennent principalement :

\begin{itemize}
    \item utilsation de \item{ray-tracing} ;
    \item utilisation de systèmes source-image.
\end{itemize}

Les deux sont connues depuis longtemps et éprouvés, elles peuvent parallèlement être améliorer de la prise en compte de
divers facteurs :

\begin{itemize}
    \item diffusion (aléatoire ou déterminée)
    \item absorption
\end{itemize}

