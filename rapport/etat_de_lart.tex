\chapter{Etat de l'art}

Les premiers essais se rapportant à l'auralisation ont été faits par Spandöck et coll. en 1929.
Ces travaux ont été repris et améliorés après l'apparition des ordinateurs ; en utilisant cette nouvelle puissance de
calcul et, vers la fin des années 1960, le premier logiciel de simulation d'acoustique des salles fut développé
(Krokstad)~\cite{Vor08}.

Le mot «auralisation» lui-même fut utilisé pour la première fois par Kleiner et coll. dans l'article
\underline{Auralization -- An Overview}~\cite{Kle93}.

Dans la sommes des techniques utilisées pour aboutir à la reproduction des conditions acoustiques d'une salle, deux
reviennent principalement :

\begin{itemize}
    \item utilsation de \item{ray-tracing} ;
    \item utilisation de systèmes source-image.
\end{itemize}

Les deux sont connues depuis longtemps et éprouvés, elles peuvent parallèlement être améliorer de la prise en compte de
divers facteurs :

\begin{itemize}
    \item diffusion (aléatoire ou déterminée)
    \item absorption
\end{itemize}

Les applications possibles de l'auralisation au terme général sont nombreuses et variées. La plus évidentes d'entre
elles est certainement la reproduction de «l'acoustique» d'une salle, mais on peut aller plus loin. L'acoustique
prédictive permet de simuler le rendu de salles et plus généralement d'espaces inexistants en combinant des mesures
entre elles ou purement par le calcul. Enfin, l'auralisation peut être utilisée dans la conception de systèmes de
réalité virtuelle à forte immersion.

Il faut enfin savoir que la complexité de rendu d'une auralisation rend difficile sa mise en place à grande échelle. Si
la projection en 3D et à 360 degrés est aujourd'hui possible \textit{via} divers processus de visualisation (le pendant
visuel de l'auralisation), l'inclusion d'un environnement acoustique pleinement contrôlé est extrèmement complexe et
demanderait un nombre impressionnant de haut parleurs (et ne pourrait s'adapter à chaque spectateur). Une restitution du
champ acoustique de manière individuelle peut être en revanche envisageable au travers de casques et d'un système de
suivi d'orientation 3D. Ce type d'environnement d'immersion existe soit en version prototypée (en Allemagne par exemple,
à Ilmenau) ou bien en version commerciale \textit{via} le projet
CAVE\textsuperscript{TM}\footnote{CAVE\textsuperscript{\textsc{TM}} :
    CAVE\textsuperscript{\textsc{TM}} Automatic Virtual
Environment}
