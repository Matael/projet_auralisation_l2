\chapter*{Conclusion}

La mise en place d'une chaîne d'auralisation (acquisition, convolution et restitution au casque) est un procédé
relativement simple et permet des résultats honorables.
Les problèmes se présentent lorsqu'une recherche d'optimisation est inclue. Si l'utilisation de RI binaurales améliore
grandement la qualité de l'auralisation (en permettant notamment un meilleur repérage spatial), ce n'est pas le seul
levier vers un rendu optimisé. L'utilisation d'autre sources qu'une simple baudruches ainsi que d'autres procédés de
mesure de RI sont autant de moyen d'aboutir à une auralisation de meilleure qualité.
L'étude des salles elles-mêmes n'a pas été traité dans ce rapport mais pourrait présenter un intéret pour choisir les
sources en fonction des salles ciblées, de même une optimisation au regard du type de sons à utiliser ensuite
permettrait de réduire les temps de calculs.

L'utilisation de procédés tels que celui-ci dans des applications de réalité virtuelle ou la qualité de l'immersion se
doit d'être la plus précise possible est aussi un vecteur de développement. Au cours du projet et notamment de l'analyse
de résultats, nous avons noté combien les ressources (processeur, mémoire) nécessaire pouvait croître rapidement. De ce
point de vue, l'écriture et la conception de systèmes dédiés à cette problématique semblent intéressantes.

Nos résultats, s'ils sont loin d'être parfaits, montrent toutefois la faisabilité de l'auralisation et que celle-ci
repose sur un ensemble de concepts assez simples.
Aucun des tests réalisés ne l'a été sur une machine dédiée à cette application, pourtant, les résultats ne nous ont
jamais semblés incohérents d'un point de vue perceptif.

Enfin, toutes les personnes ayant écoutés les résultats d'auralisations les ont trouvés plutôt représentatifs. La salle
réverbérante a toujours été bien reconnue et le repérage de la source dans l'espace s'est fait sans souci.
Aucun test n'a montré une particulièrement bonne estimation de la distance entre source et point d'écoute.
A noter enfin que tous ceux ayant écouté les sons résultant, nous compris, ont été suppris du réalisme de certains signaux.

L'auralisation est un champ d'application relativement récent. Même si les premières recherches sur ce thème remontent à
plus de 70ans, beaucoup de choses restent à faire.
L'utilisation de telles techniques hors des travaux de laboratoire demandera encore beaucoup de travail pour que
l'immersion de l'auditoire soit la plus complète possible.
