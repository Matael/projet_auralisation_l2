\chapter*{Introduction}

Pour le public, l'acoustique est un domaine s'appliquant principalement aux salles.
Bien que la majorité pense à l'amélioration des performances acoustiques, l'étude réelle autour des salles va
beaucoup plus loin.
La reproduction des conditions d'écoute dans une salle donnée (réelle ou virtuelle) est un sujet important. Il s'agit d'une
application à la frontière entre acoustique des salles et réalité virtuelle, le tout teinté de psychoacoustique. Dans les
domaines s'approchant, on citera notament la reproduction de transducteurs (en captation ou reproduction).

\medskip

Le fait de recréer la modification d'un son par une salle à partir de mesures ou de calculs s'appelle
l'\emph{auralisation}. L'auralisation est d'ailleurs définie ainsi dans l'ouvrage de Michael
Vorländer~\cite{Vor08} :

\begin{quote}
L'auralisation est une technique visant à créer des fichiers sonores écoutables depuis des données (simulées, mesurées
ou synthétisées) numériques.
\end{quote}

\medskip

Afin de mettre en œuvre une comparaison de l'influence de différents facteurs sur la qualité d'une auralisation, une
série de mesures est effectuée (réponses impulsionnelles -- RI --  binaurales et monaurales, sons en salles cibles,
etc...).  Ensuite, les signaux mesurés sont convolués avec les RI et le résultat est écouté et qualifié. Finalement, la
comparaison même prend forme et les résultats sont consignés et interprétés : il s'agit alors de comparer le résultat
obtenu par convolution «\emph{mathématique}» avec le rendu réel (convolution «\emph{physique}» en rejouant le son en
salle cible) et ce en variant divers paramètres (RI monaurale/binaurale, mode de convolution, etc...).
