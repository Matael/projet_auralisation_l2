\chapter*{Introduction}
Pour le public, l'acoustique est un domaine s'appliquant principalement aux salles.
Bien que la majorité pense à l'amélioration des performances acoustiques, l'étude réelle autour des salles va
beaucoup plus loin.
La reproduction des conditions d'écoute dans une salle donnée (réelle ou virtuelle) est un sujet important. Il s'agit d'une
application à la frontière entre acoustique des salles et réalité virtuelle, le tout teinté de psychoacoustique. Dans les
domaines s'approchant, on citera notament la reproduction de transducteurs (en captation ou reproduction) et la modélisation de son dits "en 3 dimensions".

\medskip

Le fait de recréer la modification d'un son par une salle à partir de mesures ou de calculs s'appelle
l'\emph{auralisation}. L'auralisation est d'ailleurs définie ainsi dans l'ouvrage de Michael
Vorländer~\cite{Vor08} :

\begin{quote}
L'auralisation est une technique visant à créer des fichiers sonores écoutables depuis des données (simulées, mesurées
ou synthétisées) numériques.

\end{quote}

\medskip

Afin de mieux saisir le principe et l'importance des différents facteurs en jeu dans le processus d'auralisation, un bref historique est dressé puis une étude théorique des outils mathématiques est menée ; celle ci conduit à la première auralisation pratique présentée. Divers essais préliminaires sont consignés ensuite avant de discuter l'influence que la source a sur le procédé, une tentative de compensation est aussi décrite. Finalement, les résultats de deux auralisations sont discutés ; cette dernière partie accueillant aussi diverses remarques sur la différence entre réponses impulsionnelles -- RI -- binaurales (deux points de captation de part et
d'autre d'une tête artificielle) et monaurales (un unique micro pour la captation).

\bigskip
\bigskip

\section*{Sources et données du projet}

Toutes les sources du projet (\Matlab{}, \LaTeX{},, etc...) ainsi que les données de mesures et les résultats sont
dsponibles en ligne :

\begin{center}
\url{https://github.com/Matael/projet_auralisation_l2}
\end{center}
